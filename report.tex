\documentclass[a4paper, 11pt]{report}
\usepackage{preamble}

\begin{document}\maketitle

\tableofcontents

\chapter{Introduction}

\chapter{The double flag variety approach to q-Schur algebras}

\chapter{The cyclic flags approach to affine q-Schur algebras}

Fix natural numbers $n$ and $r$.

\begin{definition}[compositions]\label{def:compositions}
A composition of $r$ into $n$ parts is an $n$-tuple $\lambda=(\lambda_1,\ldots,\lambda_n)\in\integers^n$ of non-negative integers whose sum equals $r$. Denote the set of compositions of $r$ into $n$ parts by $\compositions$.
\end{definition}

\begin{definition}[infinite periodic matrices]\label{def:matrices}
Let $\matrices$ be the set of matrices $A=(a_{i,j})_{i,j\in\integers}$ with integer entries $a_{i,j}$ satisfying the following conditions: 
\begin{itemize}
\item
$a_{i,j}\geq 0$ for each $i,j\in\integers$;
\item
each row or column has only finitely many non-zero entries;
\item
the sum of the entries in any $n$ consecutive rows or columns equals $r$;
\item
$a_{i-n,j-n}=a_{i,j}$ for each $i,j\in\integers$.
\end{itemize}
These matrices are referred to as infinite periodic matrices.
\end{definition}

\begin{definition}[source and target]\label{def:source-target}
Given $A\in\matrices$, let $\ro{A}$ and $\co{A}$ be the compositions of $r$ into $n$ parts given by
\begin{equation*}
\ro{A} = \left( \sum_{j\in\integers}a_{1,j}, \ldots, \sum_{j\in\integers} a_{n,j}\right)
\end{equation*}
and
\begin{equation*}
\co{A} = \left( \sum_{i\in\integers} a_{i,1}, \ldots, \sum_{i\in\integers} a_{i,n}\right).
\end{equation*}
$A\in\matrices$ is said to go from $\co{A}$ to $\ro{A}$.
\end{definition}

\begin{definition}[diagonal matrices]
Given $\lambda\in\compositions$, let $D_\lambda\in\matrices$ be the matrix given by $(D_\lambda)_{i,j}=0$ for $i,j\in\integers$ with $i\neq j$ and $(D_\lambda)_{i,i}=\lambda_i$ for $i\in\integers$; where the indices are taken modulo $n$.
\end{definition}

\section{Cyclic flags}

Fix $n,r\in\naturals$ and let $\field$ be a field. Let $\laurent$ be the $\field$-algebra $\field[\epsilon,\epsilon^{-1}]$ and let $\polys$ be the subalgebra generated by $\epsilon$, so $\polys = \field[\epsilon]$. Let $V$ be a free $\laurent$-module of rank $r$. Let $G$ be the automorphism group of the $\laurent$-module $V$, so $G$ is isomorphic to $\GL{r}{\laurent}$. A lattice in $V$ is a $\polys$-submodule $L$ of $V$ with $\laurent\otimes_\polys L = V$. In particular, a lattice is an $\polys$-submodule of $V$ which is a free $\polys$-module of rank $r$. Let $\flags=\flags_\field(n,r)$ be the set of collections $(L_i)_{i\in\integers}$ of lattices in $V$ with $L_i\subset L_{i+1}$ and $\epsilon L_i = L_{i-n}$ for each $i\in\integers$. These collections of lattices in $V$ are referred to as cyclic flags in $V$. 

$G$ acts on $\flags$ by $(g\cdot L)_i = g(L_i)$ for each $i\in\integers$, given $g\in G$ and $L\in\flags$. The $G$-orbits in $\flags$ are indexed by the set $\compositions$ of compositions of $r$ into $n$ parts: the $G$-orbit in $\flags$ corresponding to $\lambda\in\compositions$ is
\begin{equation*}
\flags_\lambda = \left\{L\in\flags: \dim\left(\frac{L_i}{L_{i-1}}\right) = \lambda_i \text{ for each } i\in\integers\right\}
\end{equation*}

\begin{definition}\label{def:characteristic-matrix}
The periodic characteristic matrix of a pair of cyclic flags $(L,L')\in\dblflags$ is the matrix $\type{L}{L'}=(a_{i,j})_{i,j\in\integers}$ with entries
\begin{equation*}
a_{i,j} = \dim_\field\left(\frac{L_i\cap L_j'}{L_i\cap L_{j-1}' + L_{i-1}\cap L_j'}\right)
\end{equation*}
for each $i,j\in\integers$.
\end{definition}

The diagonal action of $G$ on $\dblflags$ has orbits indexed by the set $\matrices$ of infinite periodic matrices (see definition \ref{def:matrices}). The $G$-orbit corresponding to $A\in\matrices$ is denoted $\mathcal{O}_A$ and consists of those pairs $(L,L')\in\dblflags$ with periodic characteristic matrix $\type{L}{L'}$ equal to $A$.

\begin{lemma}(alternative expression for characteristic matrix)
Alternatively,
\begin{equation*}
a_{i,j} = \dim_\field\left(\frac{L_{i-1} + L_i\cap L_j'}{L_{i-1} + L_i\cap L_{j-1}'}\right),
\end{equation*}
for each $i,j\in\integers$.
\end{lemma}
\begin{proof}
Set $U=L_i\cap L_j'$ and $U'=L_{i-1}+L_i\cap L_{j-1}'$. Then $U+U'=L_{i-1}+L_i\cap L_j'$ and $U\cap U'= L_i\cap L_j'\cap L_{i-1} + L_i\cap L_{j-1}'$. Applying the isomorphism theorems, ${U+U'}/{U'}$ is naturally isomorphic to $U/{U\cap U'}$ as a vector space. In particular,
\begin{equation*}
\frac{L_{i-1}+L_i\cap L_j'}{L_{i-1} + L_i\cap L_{j-1}'} = \frac{L_i\cap L_j'}{L_{i-1}\cap L_j' + L_i\cap L_{j-1}'}
\end{equation*}
and thus the dimensions of these spaces are both equal to $a_{i,j}$.
\end{proof}

\begin{lemma}[transposing characteristic matrix]
Given a pair of flags $(L,L')\in\flags^2$, the matrices $\type{L}{L'}$ and $\type{L'}{L}$ are related by the transpose. In particular, $\type{L}{L'}_{i,j} = \type{L'}{L}_{j,i}$ for each $i,j\in\integers$.
\end{lemma}
\begin{proof}
By swapping the roles of $i$ and $j$ and swapping $L$ and $L'$ it is clear that $\type{L}{L'}_{i,j}$ and $\type{L'}{L}_{j,i}$ are both given by the dimension of the $\field$-vector space
\begin{equation*}
\frac{L_i\cap L_j'}{L_{i-1}\cap L_j' + L_i\cap L_{j-1}'},
\end{equation*}
for each $i,j\in\integers$.
\end{proof}

\begin{lemma}[a codimension formula]\label{lemma:flags-codimension-formula}
Given $(L,L')\in\flags^2$ and $i,j\in\integers$,
\begin{equation*}
\dim_\field\left(\frac{L_i}{L_i\cap L_j'}\right) = \sum_{s\le i, t>j} a_{s,t},
\end{equation*}
where $\type{L}{L'}=(a_{i,j})_{i,j\in\integers}$.
\end{lemma}
\begin{proof}
{\color{red}COMPLETE THIS PROOF}	
\end{proof}

\begin{lemma}[nested flags]
Given $(L,L')\in\flags^2$, $L'\subset L$ if and only if $\type{L}{L'}_{i,j}=0$ for $i,j\in\integers$ with $i>j$.
\end{lemma}
\begin{proof}
Suppose $L,L'\in\flags$ with $L'\subset L$, meaning $L_j'\subset L_j$ for each $j\in\integers$. Then for $i>j$, $L_i\cap L_j' = L_j'$, $L_{i-1}\cap L_j' = L_j'$ and $L_i\cap L_{j-1}'$, which shows
\begin{equation*}
\type{L}{L'}_{i,j} = \dim_\field\left(\frac{L_j'}{L_{j-1}'+L_j'}\right) = 0
\end{equation*}
as required. Conversely, suppose $\type{L}{L'}$ is upper triangular, meaning $\type{L}{L'}_{i,j}=0$ when $i>j$. Using Lemma \ref{lemma:flags-codimension-formula},
\begin{equation*}
\dim_\field\left(\frac{L_i'}{L_i'\cap L_i}\right) = \sum_{s>i,t\le i} a_{s,t} = 0,
\end{equation*}
so $L_i\cap L_i' = L_i'$ and thus $L_i'\subset L_i$ for each $i\in\integers$, as required.
\end{proof}

\begin{corollary}[diagonal orbits]\label{corollary:diagonal-orbits}
Given $L,L'\in\flags$, $L=L'$ if and only if $\type{L}{L'}_{i,j}=0$ whenever $i\neq j$. In particular,
\begin{equation*}
\mathcal{O}_{D_\lambda} = \{(L,L)\in\flags^2: L\in\flags_\lambda\},
\end{equation*}
for each $\lambda\in\compositions$.
\end{corollary}

Given $A,B\in\matrices$ with $\co{A}=\ro{B}$, define
\begin{equation*}
\yprod{A,B} = \{ (L,L',L'')\in\flags^3: (L,L')\in\mathcal{O}_A \text{ and } (L',L'')\in\mathcal{O}_B\},
\end{equation*}
\begin{equation*}
\xprod{A,B} = \{(L,L'')\in\flags^2:\exists L'\in\flags\text{ with } (L,L')\in\mathcal{O}_A \text{ and } (L',L'')\in\mathcal{O}_B\}.
\end{equation*}
If also $L\in\flags_{\ro{A}}$, define the $L$-slices of $\yprod{A,B}$ and $\xprod{A,B}$ respectively as
\begin{equation*}
\yprod[L]{A,B} = \{(L',L'')\in\flags^2: (L,L',L'')\in\yprod{A,B}\},
\end{equation*}
\begin{equation*}
\xprod[L]{A,B} = \{L''\in\flags: (L,L'')\in\xprod{A,B}\}.
\end{equation*}

\begin{observation}
There are only finitely many $G$-orbits in $\xprod{A,B}$.
\end{observation}

\begin{lemma}\label{lemma:product-with-diagonal-orbits}
Given $A\in\matrices$, $\xprod{D_\lambda,A}=\mathcal{O}_A$ if $\lambda=\ro{A}$ and $\xprod{A,D_\lambda}=\mathcal{O}_A$ if $\lambda=\co{A}$.
\end{lemma}

\begin{proof}
Let $A\in\matrices$ and set $\lambda=\ro{A}$. $\yprod{D_\lambda,A}$ is the set of triples $(L,L',L'')\in\flags^3$ with $(L,L')\in\mathcal{O}_{D_\lambda}$, thus $L=L'$ by Corollary \ref{corollary:diagonal-orbits}, and $(L',L'')\in\mathcal{O}_A$. $\xprod{D_\lambda,A}$ is the projection of $\yprod{D_\lambda,A}$, which equals $\mathcal{O}_A$.

Similarly, if $\lambda=\co{A}$, $\yprod{A,D_\lambda}$ is the set of triples $(L,L',L'')\in\flags^3$ with $(L,L')\in\mathcal{O}_A$ and $L''=L'$, so $\xprod{A,D_\lambda}$ is exactly the orbit $\mathcal{O}_B$.
\end{proof}

\subsection{Triple products}

Given $A,B,C\in\matrices$ with $\co{A}=\ro{B}$ and $\co{B}=\ro{C}$ and $L\in\flags_{\ro{A}}$, there are spaces $\xprod{A,B,C}$, $\yprod{A,B,C}$ and their respective $L$-slices, defined as follows:
\begin{equation*}
\yprod{A,B,C} = \{(L,L',L'',L''')\in\flags^4: (L,L')\in\mathcal{O}_A, (L',L'')\in\mathcal{O}_B \text{ and } (L'',L''')\in\mathcal{O}_C\},
\end{equation*}
\begin{equation*}
\xprod{A,B,C} = \{(L,L''')\in\flags^2: \exists (L',L'')\in\mathcal{O}_B \text{ with } (L,L')\in\mathcal{O}_A \text{ and } (L'',L''')\in\mathcal{O}_C\},
\end{equation*}
\begin{equation*}
\yprod[L]{A,B,C} = \{ (L',L'',L''')\in\flags^3: (L,L',L'',L''')\in\yprod{A,B,C}\},
\end{equation*}
\begin{equation*}
\xprod[L]{A,B,C} = \{ L'''\in\flags: (L,L''')\in\xprod{A,B,C}\}.
\end{equation*}

\section{Convolution algebras}

Suppose $\field$ is a finite field and let $\mathrm{q}$ denote the number of elements of $\field$. Consider the set $S$ of $G$-invariant functions $\dblflags\to\integers$ with constructible support. $S$ is a free $\integers$-module with a basis consisting of the indicator functions of the $G$-orbits in $\dblflags$. Define an operation $\star$ on $S$ as follows: for each $f,g\in S$, $f\star g\in S$ is given by
\begin{equation*}
(f\star g)(L,L'') = \sum_{L'\in\flags} f(L,L')g(L',L''),
\end{equation*}
for $(L,L'')\in\dblflags$. 

$f\star g$ is well defined since the supports of $f$ and $g$ consist of finitely many $G$-orbits, so there are only finitely many $L'\in\flags$ such that $f(L,L')g(L',L'')\neq 0$, given $(L,L'')\in\dblflags$. $f\star g$ is constant on $G$-orbits and is supported on finitely many $G$-orbits, so $f\star g\in S$.

\begin{lemma}\label{lemma:convolution-algebra}
The set $S$ together with the operation $\star$ is an associative $\integers$-algebra with identity element $\iota$ given by $\iota(L,L) = 1$ and $\iota(L,L')=0$ for $L'\neq L$.
\end{lemma}

\begin{proof}
Given $f,g,h\in S$ and $(L,L''')\in\dblflags$,
\begin{align*}
((f\ast g)\star h)(L,L''')
&= \sum_{L''} (f\star g)(L,L'')h(L'',L''')\\
&= \sum_{L''}\sum_{L'} f(L,L')g(L',L'')h(L'',L''')\\
&= (f\star (g\star h))(L,L'''),
\end{align*}
thus $\star$ is associative. $\iota$ is the multiplicative identity since
\begin{equation*}
(\iota\star f)(L,L'') = \sum_{L'}\iota(L,L')f(L',L'') = f(L,L'')
\end{equation*}
and
\begin{equation*}
(f\star\iota)(L,L'') = \sum_{L'}f(L,L')\iota(L',L'') = f(L,L''),
\end{equation*}
for each $f\in S$ and $(L,L'')\in\dblflags$.
\end{proof}

Given $A\in\matrices$, let $e_A\in S$ denote the indicator function of the orbit $\mathcal{O}_A$. $S$ is a free $\integers$-module with basis $\{e_A:A\in\matrices\}$. There exist $\gamma_{A,B,C;\mathrm{q}}\in\integers$ for $A,B,C\in\matrices$ such that
\begin{equation*}
e_A\star e_B = \sum_{C\in\matrices} \gamma_{A,B,C;\mathrm{q}}e_C
\end{equation*}
for each $A,B\in\matrices$. Then
\begin{align*}
\gamma_{A,B,C;\mathrm{q}}
&=(e_A\star e_B)(L,L'')\\
&= \sum_{L'} e_A(L,L')e_B(L',L'')\\
&= \#\{L':(L,L')\in\mathcal{O}_A \text{ and }(L',L'')\in\mathcal{O}_B\},
\end{align*}
for any $(L,L'')\in\mathcal{O}_C$.

\section{Affine q-Schur algebras}

There exist polynomials $\gamma_{A,B,C}\in\integers[q]$ for $A,B,C\in\matrices$ such that $\gamma_{A,B,C}(\mathrm{q}) = \gamma_{A,B,C;\mathrm{q}}$ for any prime power $\mathrm{q}$, following \cite[section 4]{lusztig99}. The affine $q$-Schur algebra $\qschur$ (defined in \needsreference) is a $\integers[q]$-algebra which is a free $\integers[q]$-module with basis $\{e_A:A\in\matrices\}$ and with multiplication given by
\begin{equation*}
e_A e_B = \sum_{C} \gamma_{A,B,C}e_C.
\end{equation*}

Given the existence of these `universal polynomials' $\gamma_{A,B,C}\in\integers[q]$, it follows from Lemma \ref{lemma:convolution-algebra} that $\qschur$ is an associative $\integers[q]$-algebra with multiplicative identity given by
\begin{equation*}
1 = \sum_{\lambda\in\compositions} e_{D_\lambda}.
\end{equation*}

\begin{lemma}
Transposition gives a homomorphism of $\integers[q]$-modules $\transpose\colon\qschur\to\qschur$ with $\transpose(e_A) = e_{A^\transpose}$, $\transpose\circ\transpose = 1$ and $\transpose(e_A e_B) = \transpose(e_B)\transpose(e_A)$.
\end{lemma}
\begin{proof}
Let $A,B,C\in\matrices$ and let $\field$ be a finite field with $\mathrm{q}=\#\field$ elements. If $(L,L'')\in\mathcal{O}_C$ then $(L'',L)\in\mathcal{O}_{C^\transpose}$ and
\begin{align*}
\gamma_{A,B,C;\mathrm{q}}
&= \#\{L': (L,L')\in\mathcal{O}_A\text{ and } (L',L'')\in\mathcal{O}_B\}\\
&= \#\{L': (L'',L')\in\mathcal{O}_{B^\transpose}\text{ and } (L',L)\in\mathcal{O}_{A^\transpose}\}\\
&= \gamma_{B^\transpose,A^\transpose, C^\transpose;\mathrm{q}}
\end{align*}
It then follows that $\transpose(e_A e_B) = \transpose(e_B) \transpose(e_A)$.
\end{proof}

\chapter{Quivers with relations for affine q-Schur algebras}

\section{Basic results: TO BE REPLACED WITH A MORE INFORMATIVE NAME.}

If $i,j\in\integers$, let $\mathcal{E}_{i,j}$ denote the `elementary matrix' with entries given by $(\mathcal{E}_{i,j})_{s,t}=1$, for $s,t\in\integers$,  whenever $(i,j)\sim (s,t)$ modulo $(n,n)$ and all other entries are zero.

Given $\lambda\in\compositions$, let $D_\lambda\in\matrices$ denote the diagonal matrix with $r(D_\lambda)=c(D_\lambda) = \lambda$. That is,
\begin{equation*}
D_\lambda = \lambda_1 \mathcal{E}_{1,1} +\cdots + \lambda_n \mathcal{E}_{n,n}
\end{equation*}

For $\lambda\in\compositions$, write $1_\lambda = e_{D_\lambda}$. The $1_\lambda$ are pairwise orthogonal idempotents in $\qschur$ with $1 = \sum_{\lambda\in\compositions}1_\lambda$.

Given $i,j\in\integers$, write $X_{i,j} = \mathcal{E}_{i,j} - \mathcal{E}_{i+1,j}$. By convention, $e_A = 0$ unless $A\in\matrices$.

For $i\in [1,n]$ and $\lambda\in\compositions$, write
\begin{equation*}
E_{i,\lambda} = e_{D_\lambda + X_{i,i+1}},
\end{equation*}
\begin{equation*}
F_{i,\lambda} = e_{D_\lambda - X_{i,i}}.
\end{equation*}

Define
\begin{equation*}
E_i = \sum_{\lambda\in\compositions} E_{i,\lambda}
\end{equation*}
\begin{equation*}
F_i = \sum_{\lambda\in\compositions} F_{i,\lambda}.
\end{equation*}

Observe that $E_{i,\lambda}=0$ unless $\lambda_{i+1} > 0$ and $F_{i,\lambda}=0$ unless $\lambda_i > 0$. Also, $E_{i,\lambda} = E_i 1_\lambda$ and $F_{i,\lambda} = F_i 1_\lambda$. 
\begin{lemma}
Let $i\in [1,n]$ and $A\in\matrices$.
\begin{equation*}
E_ie_A = \sum_{p\in\integers} q^{\sum_{j>p} a_{i,j}}[a_{i,p}+1] e_{A + X_{i,p}}
\end{equation*}
and
\begin{equation*}                                                                               
F_ie_A = \sum_{p\in\integers} q^{\sum_{j<p} a_{i+1,j}}[a_{i+1,p}+1] e_{A - X_{i,p}}.
\end{equation*}
\end{lemma}

Note that these formulas are still valid in the cases $E_ie_A=0$ and $F_ie_A=0$. There are similar formulas for right multiplication by $E_i$ and $F_i$, which can be obtained by applying the transpose involution to the above formulas. The transpose relates the $E_i$, $F_i$ and $1_\lambda$ in the following way: $\transpose(E_{i,\lambda}) = F_{i,\lambda}$, $\transpose(F_{i,\lambda}) = E_{i,\lambda -\epsilon_i +\epsilon_{i+1}}$ and $\transpose(1_\lambda) = 1_\lambda$. In particular, $\transpose(E_i) = F_i$ and $\transpose(F_i) = E_i$.

\begin{corollary}
Let $j\in [1,n]$ and $A\in\matrices$. Then
\begin{equation*}
e_A F_j = \sum_{p\in\integers} q^{\sum_{i>p} a_{i,j}}[a_{p,j}+1] e_{A+X_{j,p}^{\transpose}}
\end{equation*}
and
\begin{equation*}
e_A E_j = \sum_{p\in\integers} q^{\sum_{i<p} a_{i,j+1}}[a_{p,j+1}+1] e_{A-X_{j,p}^{\transpose}}
\end{equation*}
\end{corollary}
\begin{proof}
\begin{align*}
e_A F_j &= \transpose (E_j e_{A^\transpose})\\
&= \transpose ( \sum_p q^{\sum_{i>p} a_{i,j}}[a_{p,j}+1] e_{A^\transpose + X_{j,p}} )\\
&= \sum_p q^{\sum_{i>p} a_{i,j}} [a_{p,j}+1] e_{A+X_{j,p}^{\transpose}}
\end{align*}

\begin{align*}
e_A E_j &= \transpose (F_j e_{A^\transpose})\\
&= \transpose ( \sum_p q^{\sum_{i<p} a_{i,j+1}} [a_{p,j+1}+1] e_{A^\transpose - X_{j,p}} )\\
&= \sum_p q^{\sum_{i<p} a_{i,j+1}}[a_{p,j+1}+1] e_{A-X_{j,p}^{\transpose}}
\end{align*}
\end{proof}

Note that $E_i^{r+1}=F_i^{r+1}=0$ while
\begin{equation*}
E_i^r = [r]_! e_{r\mathcal{E}_{i,i+1}}
\end{equation*}
and
\begin{equation*}
F_i^r = [r]_! e_{r\mathcal{E}_{i+1,i}}.
\end{equation*}

\begin{lemma}[quantum Serre relations: $n\geq 3$]
Suppose $n\geq 3$. The following relations hold in $\qschur$:
\begin{equation*}
E_i E_j - E_j E_i = 0
\end{equation*}
\begin{equation*}
F_i F_j - F_j F_i = 0
\end{equation*}
unless $j=i\pm 1$;
\begin{align*}
E_i E_{i+1}^2 - (1+q)E_{i+1} E_i E_{i+1} + qE_{i+1}^2 E_i &= 0\\
E_i^2 E_{i+1} - (1+q)E_i E_{i+1} E_i + qE_{i+1} E_i^2 &= 0
\end{align*}
and
\begin{align*}
F_{i+1} F_i^2 - (1+q) F_i F_{i+1} F_i + qF_i^2 F_{i+1} &=0\\
F_{i+1}^2 F_i - (1+q) F_{i+1} F_i F_{i+1} + q F_i F_{i+1}^2 &= 0.
\end{align*}
\end{lemma}
\begin{proof}
Here we introduce temporary notation for the basis elements: Write $[ A] = e_A$.

% first relation for Es:
Take $\lambda\in\compositions$.
\begin{equation*}
E_i E_{i+1}^2 1_\lambda = [2][ {D_\lambda + 2X_{i+1,i+2} + X_{i,i+2} }] + [2][ { D_\lambda + 2X_{i+1,i+2} + X_{i,i+1} }]
\end{equation*}                                                                                                                                                                                                                                                                                                                                      

\begin{equation*}
E_{i+1} E_i E_{i+1} 1_\lambda = [ {D_\lambda + 2X_{i+1,i+2} + X_{i,i+1} } ] + [2][ { D_\lambda + 2X_{i+1,i+1} + X_{i,i+1} } ]
\end{equation*}

\begin{equation*}
E_{i+1}^2 E_i 1_\lambda = [2][ { D_\lambda + 2X_{i+1,i+2} + X_{i,i+1} } ]
\end{equation*}
Then
\begin{equation*}
(E_i E_{i+1}^2 - (1+q)E_{i+1} E_i E_{i+1} + qE_{i+1}^2 E_i)1_\lambda = 0,
\end{equation*}
for each $\lambda\in\compositions$. The relation $E_i E_{i+1}^2 - (1+q)E_{i+1} E_i E_{i+1} + qE_{i+1}^2 E_i = 0$ then follows.

% Second relation for Es:
% ...
% ...

The relations between $F_i$ and $F_{i+1}$ may be obtained directly, as above, or by applying the transpose operator to the relations already derived: note that the two sets of relations are related by swapping $E_i$ and $F_i$ and reversing the order of multiplication.
\end{proof}



\begin{lemma}[quantum Serre relations: $n=2$]
{\color{gray} In the case $n=2$, the quantum Serre relations will be of total degree $4$. Look at the presentation of quantum groups for candidate relations. If that fails, brute force won't be too hard.}
\end{lemma}

\begin{lemma}
$[E_i, F_j] = 0$ unless $j=i$.
\begin{equation*}
E_i F_i - F_i E_i = \sum_{\lambda\in\compositions} ([\lambda_i] - [\lambda_{i+1}])1_\lambda.
\end{equation*}
\end{lemma}


{\color{gray}
For $\lambda\in\compositions$, let $R_\lambda = e_{\lambda_1\mathcal{E}_{0,1}+\cdots +\lambda_n\mathcal{E}_{n-1,n}}$. Write $R = \sum_{\lambda\in\compositions} R_\lambda$. Note $R_\lambda = R 1_\lambda$. Given $A\in\matrices$ and $m\in\integers$, let $A[m]\in\matrices$ be given by $A[m]_{i,j} = a_{i,j+m}$ and let $A^{[m]}$ be given by $A^{[m]}_{i,j} = a_{i+m,j}$ for each $i\in\integers$.

\begin{lemma}[Shifting]
If $A\in\matrices$ then
\begin{equation*}
R e_A = e_{A^{[\pm 1]}}
\end{equation*}
and
\begin{equation*}
e_A R = e_{A_{[\pm 1]}}.
\end{equation*}
\end{lemma}

Conjugation by $R$ gives an automorphism $\rho$ of $\qschur$ satisfying $\rho^n = 1$.
}

\section{quivers with relations}

Denote by $\compositions$ the set of compositions of $r$ into $n$ parts. That is, $\compositions$ is the set of $\alpha\in\integers^n$ with non-negative entries which sum to $r$. Let $\epsilon_i\in\integers^n$ be the $i$th elementary vector and write $\alpha_i = \epsilon_i  -\epsilon_{i+1}$ for each $i\in [1,n]$. Then $\lambda +\alpha_i\in\compositions$ if $\lambda_{i+1}>0$ and $\lambda -\alpha_i\in\compositions$ if $\lambda_i >0$.

Let $\Gamma=\presentationquiver$ be the quiver with set of vertices $\compositions$, with the following arrows:

For $\lambda\in\compositions$ and $i\in [1,n]$, there is an arrow $e_{i,\lambda}:\lambda\to\lambda +\alpha_i$ if $\lambda_{i+1}>0$ and there is an arrow $f_{i,\lambda}:\lambda\to\lambda -\alpha_i$ if $\lambda_i>0$.

Denote by $\integers[q]\Gamma$ the path $\integers[q]$-algebra of $\Gamma$. Thus $\integers[q]\Gamma$ is a free $\integers[q]$-module with a basis given by the set of paths in $\Gamma$, with multiplication given by the concatenation of paths. If $p$ starts where $q$ ends, the product $pq$ is the path $q$ followed by $p$. Write $e_{i,\lambda}=0$ unless $\lambda,\lambda+\alpha_i\in\compositions$ and write $f_{i,\lambda}=0$ unless $\lambda,\lambda -\alpha_i\in\compositions$.

By construction, there is a homomorphism of $\integers[q]$-algebras
\begin{equation*}
\phi\colon \integers[q]\Gamma\to\qschur
\end{equation*}
given by
\begin{align*}
\phi(e_{i,\lambda}) &= E_{i,\lambda}\\
\phi(f_{i,\lambda}) &= F_{i,\lambda}\\
\phi(k_\lambda) &= 1_{\lambda},
\end{align*}
for $i\in [1,n]$ and $\lambda\in\compositions$.

The image of $\phi$ is the subalgebra of $\qschur$ generated by $E_i$, $F_i$ for $i\in [1,n]$ and $1_\lambda$ for $\lambda\in\compositions$, since $E_{i,\lambda}=E_i1_\lambda$ and $F_{i,\lambda}=F_i1_\lambda$, while $E_i = \sum_\lambda E_{i,\lambda}$ and $F_i = \sum_\lambda F_{i,\lambda}$. In general $\phi$ is not surjective, so this does not always lead to a presentation of $\qschur$.

\subsection{Exceptional case $n=2$.}

Describe the quiver.

Define an ideal of relations in the path algebra.

Write down the homomorphism from the bound quiver algebra to the q-Schur algebra.

\subsection{Typical case $n>2$.}

Suppose $n\geq 3$. Then $\Gamma=\presentationquiver$ has vertex set $\compositions$.\seeyoulater

Define $e_i, f_i\in\integers[q]\presentationquiver$ by
\begin{equation*}
e_i = \sum_{\lambda\in\compositions} e_{i,\lambda}
\end{equation*}
and
\begin{equation*}
f_i = \sum_{\lambda\in\compositions} f_{i,\lambda},
\end{equation*}
with the convention $e_{i,\lambda} = 0$ unless $\lambda_{i+1}>0$ and $f_{i,\lambda} = 0$ unless $\lambda_i>0$. Let $k_\lambda$ denote the constant path at vertex $\lambda$. $\{k_\lambda:\lambda\in\compositions\}$ is a set of pairwise orthogonal idempotents in $\integers[q]\presentationquiver$.

Let $\quiverrelations\subset\integers[q]\presentationquiver$ be the ideal generated by the expressions
\begin{equation*}
e_i e_{i+1}^2 -(1+q)e_{i+1}e_ie_{i+1} + qe_{i+1}^2e_i
\end{equation*}
\begin{equation*}
e_i^2e_{i+1} - (1+q) e_ie_{i+1}e_i + qe_{i+1}e_i^2
\end{equation*}
\begin{equation*}
f_{i+1}f_i^2 - (1+q)f_if_{i+1} f_i + qf_i^2f_{i+1}
\end{equation*}
\begin{equation*}
f_{i+1}^2f_i - (1+q)f_{i+1}f_if_{i+1} + qf_if_{i+1}^2
\end{equation*}
\begin{equation*}
e_if_j - f_je_i - \delta_{i,j} \sum_{\lambda\in\compositions} ([\lambda_i]-[\lambda_{i+1}])k_\lambda
\end{equation*}

Recall that a relation is a $\integers[q]$-linear combination of paths with common start and end vertices. The relations involving paths $\lambda\to\mu$ are given by $1_\mu \text{expr} 1_\lambda$, for each of the above expressions.

\begin{lemma}
There is a homomorphism of $\integers[q]$-algebras
\begin{equation*}
\phi\colon\integers[q]\presentationquiver/{\quiverrelations}\to\qschur
\end{equation*}
given by
\begin{equation*}
\phi(e_{i,\lambda}) = E_{i,\lambda}
\end{equation*}
\begin{equation*}
\phi(f_{i,\lambda}) = F_{i,\lambda}
\end{equation*}
\begin{equation*}
\phi(k_\lambda) = 1_\lambda.
\end{equation*}
\end{lemma}


\chapter{A generic affine Schur algebra}

\section{Introducing the affine generic algebra}

Assume $\field = \complex$ and fix $n,r\geq 1$. Let $\laurent$ be the $\field$-algebra $\field[\epsilon,\epsilon^{-1}]$ and let $\polys$ be the subalgebra generated by $\epsilon$, namely $\polys=\field[\epsilon]$. Let $V$ be a free $\laurent$-module of rank $r$ and let $\flags = \flags_\field(n,r)$ be the set of $n$-periodic cyclic flags in $V$; so $\flags$ consists of collections $L = (L_i)_{i\in\integers}$ of $\polys$-lattices in $V$ with $L_i\subset L_{i+1}$ for $i\in\integers$ and $\epsilon L_i = L_{i-n}$ for $i\in\integers$.

Let $G$ be the group of $\laurent$-module automorphisms of $V$. Thus $G$ is isomorphic to $\GL{r}{\laurent}$. $G$ acts on $\flags$ with orbits $\{\flags_\lambda:\lambda\in\compositions\}$, where $\compositions$ is the set of compositions of $r$ into $n$ parts, as in Definition \ref{def:compositions}.

The diagonal action of $G$ on $\dblflags$ has orbits $\{\mathcal{O}_A:A\in\matrices\}$, where $\mathcal{O}_A$ consists of those pairs of flags with periodic characteristic matrix equal to $A$. Definitions of the periodic characteristic matrix and the set $\matrices$ are given in Definition \ref{def:characteristic-matrix} and Definition \ref{def:matrices} respectively. In particular, the periodic characteristic matrix of a pair $(L,L')\in\dblflags$ is the $\integers\times\integers$ matrix $A = (a_{i,j})_{i,j\in\integers}$, with
\begin{equation*}
a_{i,j} = \dim\left(\frac{L_i\cap L_j'}{L_{i-1}\cap L_j' + L_i\cap L_{j-1}'}\right),
\end{equation*}
for each $i,j\in\integers$.

\subsection{Not quite a category}

There are maps $\ro,\co\colon\matrices\to\compositions$ given by
\begin{equation*}
\ro{A} = \left(\sum_{j\in\integers} a_{1,j},\ldots, \sum_{j\in\integers}a_{n,j}\right)
\end{equation*}
and
\begin{equation*}
\co{A} = \left(\sum_{i\in\integers} a_{i,1},\ldots, \sum_{i\in\integers} a_{i,n}\right).
\end{equation*}

Given $A\in\matrices$, write $\co{A}\stackrel{A}{\to}\ro{A}$. The purpose of this chapter is to define a category with objects $\compositions$ and morphisms $\matrices$; where $\HOM(\lambda,\mu) = \{A\in\matrices: \ro{A}=\mu,\co{A}=\lambda\}$. Given $A,B\in\matrices$ let $\matrices_{A,B}$ be the set of $C\in\matrices$ such that there exist $L,L',L''\in\flags$ with $(L,L')\in\mathcal{O}_A$, $(L',L'')\in\mathcal{O}_B$ and $(L'',L''')\in\mathcal{O}_C$. It will be shown that $\matrices$ admits a partial order $\le$ such that $\matrices_{A,B}$ has a maximum element $A\ast B$, whenever $\co{A} = \ro{B}$. It will be shown that $\ast$ is associative, so defining the composition of morphisms in the category formed by $\compositions$ and $\matrices$.

The generic affine Schur algebra $\generic$ will then be a $\integers$-algebra defined as a linearisation of this category. It will be shown that $\hat{G}(n,r)$ gives a realisation of the affine $0$-Schur algebra $\hat{S}_0(n,r)$ when $r<n$. It is expected that a more refined presentation of the generic algebra and the 0-Schur algebra will allow the conditions on the parameters to be relaxed slightly: the $r=n$ case is approachable, which may extend to the case $r<2n$.

\section{A partial order}

Given $i,j\in\integers$, define a map $d_{i,j}$ on $\matrices$ by setting
\begin{equation*}
d_{i,j}A = \sum_{s\le i,t>j} a_{s,t}
\end{equation*}
for each $A\in\matrices$.

\begin{lemma}\label{lemma:differentials}
Let $A\in\matrices$, with $A=(a_{i,j})_{i,j\in\integers}$ and write $d_{i,j} = d_{i,j}A$ for $i,j\in\integers$. Then
\begin{equation*}
d_{i,j} - d_{i-1,j} = \sum_{t>j}a_{i,t}
\end{equation*}
and
\begin{equation*}
d_{i,j}-d_{i,j-1} = - \sum_{s\le i}a_{s,j}.
\end{equation*}
\end{lemma}

\begin{proof}
Let $i,j\in\integers$. Then
\begin{equation*}
d_{i,j} - d_{i-1,j} = \sum_{s\le i,t>j}a_{s,t} - \sum_{s\le i-1,t>j}a_{s,t} = \sum_{t>j}a_{i,t}.
\end{equation*}
Similarly,
\begin{equation*}
d_{i,j}-d_{i,j-1} = \sum_{s\le i,t>j}a_{s,t} - \sum_{s\le i,t>j-1}a_{s,t} = -\sum_{s\le i}a_{s,j}.
\end{equation*}
\end{proof}

\begin{lemma}\label{lemma:antisymmetry}
Let $A\in\matrices$, with $A = (a_{i,j})_{i,j\in\integers}$ and write $d_{i,j}=d_{i,j}A$ for each $i,j\in\integers$. Then
\begin{equation*}
a_{i,j} = d_{i,j-1} - d_{i-1,j-1} + d_{i-1,j} - d_{i,j}
\end{equation*}
for each $i,j\in\integers$.
\end{lemma}
\begin{proof}
Using Lemma \ref{lemma:differentials},
\begin{align*}
a_{i,j}
&= \sum_{t>j-1}a_{i,t} - \sum_{t>j}a_{i,t}\\
&= (d_{i,j-1} - d_{i-1,j-1}) - (d_{i,j} - d_{i-1,j}).
\end{align*}
Alternatively,
\begin{align*}
a_{i,j}
&= \sum_{s\le i}a_{s,j} - \sum_{s\le i-1}a_{s,j}\\
&= -(d_{i,j}-d_{i,j-1}) + (d_{i-1,j} - d_{i-1,j-1}).
\end{align*}
\end{proof}

\begin{lemma}\label{lemma:orbit-poset}
The relation $\le$ on $\matrices$, defined by $A\le B$ if and only if $d_{i,j}A\le d_{i,j}B$ for all $i,j\in\integers$, is a partial order.
\end{lemma}

\begin{proof}
It is clear that $\le$ is reflexive and transitive, so it remains to see $\le$ is antisymmetric. Suppose $A,B\in\matrices$ with $A\le B$ and $B\le A$. Then $d_{i,j}A = d_{i,j}B$ for each $i,j\in\integers$, which shows $A=B$ as a result of Lemma \ref{lemma:antisymmetry}.  
\end{proof}

The partial order on $\matrices$ induces a partial order on the set of $G$-orbits in $\dblflags$, such that $\mathcal{O}_A\le \mathcal{O}_B$ if and only if $A\le B$. The following lemma is rephrased from Lemma \ref{lemma:flags-codimension-formula} and gives some geometric significance to the partial order on $\matrices$.

\begin{lemma}
Let $A\in\matrices$ and take $(L,L')\in\mathcal{O}_A$. Then
\begin{equation*}
d_{i,j}A = \dim\left(\frac{L_i}{L_i\cap L_j'}\right)
\end{equation*}
for each $i,j\in\integers$.
\end{lemma}
\begin{proof}
This is a rephrasing of Lemma \ref{lemma:flags-codimension-formula}.
\end{proof}

\begin{remark}
It is thought* that the partial order on $\matrices$ is compatible with the degeneration order (or closure order) on $G$-orbits in $\dblflags$ when $\field=\complex$. In particular, it is hoped that $A\le B$ if and only if $\mathcal{O}_A\subset \overline{\mathcal{O}_B}$.
\end{remark}

\section{Preliminary results}

Fix $L\in\flags$.

\begin{lemma}
$L_0/{\epsilon L_0}$ is a torsion $\field[\epsilon]$-module, where $\epsilon$ acts as zero, with dimension $r$ as a $\field$-vector space.
\end{lemma}
\begin{proof}
Let $V = \field[\epsilon,\epsilon^{-1}]^r$. $L_0$ is a free $\field[\epsilon]$-module of rank $r$, with $L_0\subset V$. So we may take a $\field[\epsilon]$-basis $x_1,\ldots, x_r\in V$ for $L_0$. The action of $\epsilon$ gives an automorphism of $V$ mapping $L_0$ to $\epsilon L_0$, so $\epsilon x_1,\ldots,\epsilon x_r$ give a basis for $\epsilon L_0$ over $\field[\epsilon]$. Therefore, the cosets $x_1 + \epsilon L_0,\ldots x_r +\epsilon L_0$ give a basis for $L_0/{\epsilon L_0}$ over $\field$.
\end{proof}

Suppose $A,B\in\matrices$ with $\co{A}=\ro{B}$. Recall the notation
\begin{equation*}
\yprod{A,B} = \{(L,L',L'')\in\flags^3: (L,L')\in\mathcal{O}_A, (L',L'')\in\mathcal{O}_B\}
\end{equation*}
and
\begin{equation*}
\xprod{A,B} = \{(L,L'')\in\flags^2:\exists L'\in\flags \text{ with } (L,L',L'')\in \yprod{A,B}\}.
\end{equation*}

$\xprod{A,B}$ is the image of $\yprod{A,B}$ under the projection onto the first and last components.

\begin{lemma}\label{lemma:orbit-products-are-bounded}
There is $N\in\naturals$ such that
\begin{equation*}
\epsilon^N L_0\subset L_0''\subset \epsilon^{-N}L_0
\end{equation*}
whenever $(L,L'')\in \xprod{A,B}$.
\end{lemma}

\begin{proof}
There exist $N_1,N_2\in\naturals$ such that
\begin{equation*}
\epsilon^{N_1}L_0\subset L_0'\subset \epsilon^{-N_1}L_0
\end{equation*}
and
\begin{equation*}
\epsilon^{N_2}L_0'\subset L_0''\subset \epsilon^{-N_2}L_0',
\end{equation*}
whenever $(L,L',L'')\in \yprod{A,B}$. Then, for $(L,L',L'')\in \yprod{A,B}$,
\begin{equation*}
L_0''\subset \epsilon^{-N_2} L_0' \subset \epsilon^{-(N_1+N_2)} L_0
\end{equation*}
and
\begin{equation*}
\epsilon^{N_1+N_2}L_0\subset \epsilon^{N_2}L_0'\subset L_0''.
\end{equation*}

In particular, taking $N=N_1 + N_2$, we have
\begin{equation*}
\epsilon^N L_0 \subset L_0'' \subset \epsilon^{-N}L_0
\end{equation*}
whenever $(L,L'')\in \xprod{A,B}$.
\end{proof}

\begin{lemma}\label{lemma:codimensions-in-orbit-product}
Suppose $N_1,N_2\in\naturals$ with $\epsilon^{N_1}L_0\subset L_0\subset \epsilon^{-N_1}L_0$ and $\epsilon^{N_2}L_0'\subset L_0''\subset \epsilon^{-N_2}L_0'$ whenever $(L,L',L'')\in \yprod{A,B}$ and let $N = N_1 + N_2$. Then
\begin{equation*}
\dim\left(\frac{\epsilon^{-N}L_0}{L_0''}\right) = d_{nN_1,0}(A) + d_{nN_2,0}(B)
\end{equation*}
and
\begin{equation*}
\dim\left(\frac{L_0''}{\epsilon^N L_0}\right) = 2Nr - d_{nN_1,0}(A) + d_{nN_2,0}(B),
\end{equation*}
whenever $(L,L'')\in \xprod{A,B}$.
\end{lemma}

\begin{proof}
Suppose $(L,L'')\in \xprod{A,B}$ and $L'\in\flags$ so that $(L,L',L'')\in \yprod{A,B}$. As in lemma \ref{lemma:orbit-products-are-bounded}, $\epsilon^N L_0\subset L_0''\subset \epsilon^{-N}L_0$, so
\begin{equation*}
\dim\left(\frac{\epsilon^{-N}L_0}{L_0''}\right) + \dim\left(\frac{L_0''}{\epsilon^N L_0}\right) = \dim\left(\frac{\epsilon^{-N}L_0}{\epsilon^N L_0}\right).
\end{equation*}

As a $\field$-vector space, $\epsilon^{-N}L_0/\epsilon^N L_0$ is isomorphic to $(L_0/{\epsilon L_0})^{2N}$, which has dimension $2Nr$, so
\begin{equation*}
\dim\left(\frac{L_0''}{\epsilon^N L_0}\right) = 2Nr - \dim\left(\frac{\epsilon^{-N}L_0}{L_0''}\right).
\end{equation*}

It remains to compute the codimension of $L_0''$ in $\epsilon^{-N}L_0$. Note $L_0''\subset \epsilon^{-N_2}L_0'\subset \epsilon^{-N} L_0$, so
\begin{equation*}
\dim\left(\frac{\epsilon{-N}L_0}{L_0''}\right) = \dim\left(\frac{\epsilon^{-N}L_0}{\epsilon^{-N_2}L_0'}\right) + \dim\left(\frac{\epsilon^{-N_2}L_0'}{L_0''}\right).
\end{equation*}

\begin{align*}
\dim\left(\frac{\epsilon^{-N}L_0}{\epsilon^{-N_2}L_0'}\right)
&= \dim\left(\frac{\epsilon^{-N_1}L_0}{L_0'}\right)\\
&= \dim\left(\frac{L_{nN_1}}{L_{nN_1}\cap L_0'}\right)\\
&= \sum_{s\le nN_1, t>0} A_{s,t}\\
&= d_{nN_1,0}(A).
\end{align*}

\begin{align*}
\dim\left(\frac{\epsilon^{-N_2}L_0'}{L_0''}\right)
&= \dim\left(\frac{ L_{nN_2}'}{L_{nN_2}'\cap L_0''}\right)\\
&= \sum_{s\le nN_2, t>0} B_{s,t}\\
&= d_{nN_2,0}(B).
\end{align*}
\end{proof}


\subsection{A quasiprojective variety}

Fix $L\in\flags$. Given $N\in\naturals$ and $\lambda\in\compositions$, define
\begin{equation*}
\Pi_{N,\lambda} = \{L''\in\flags_\lambda: \epsilon^N L_0\subset L_0''\subset \epsilon^{-N}L_0\}.
\end{equation*}
and
\begin{equation*}
\Pi_{N,\lambda}^a = \left\{L''\in\flags_\lambda: \epsilon^N L_0\subset L_0''\subset \epsilon^{N}L_0, \dim\left(\frac{\epsilon^{-N}L_0}{L_0''}\right) = a\right\}.
\end{equation*}

$\Pi_{N,\lambda}$ is the (disjoint) union of the $\Pi_{N,\lambda}^a$ for $a\in\naturals$. In fact, we will see $\Pi_{N,\lambda}^a$ is empty whenever $a > 2Nr$.

{\color{red}THE LEMMA BELOW IS NOT CORRECT}
\begin{lemma}
Let $N,a\in\naturals$, $\lambda\in\compositions$. Then $\Pi_{N,\lambda}
^a$ is nonempty exactly when $0\le a \le 2Nr$.
\end{lemma}

\begin{proof}
Suppose $L''\in\Pi_{N,\lambda}$. By definition, $\epsilon^{-N}L_0\subset L_0''\subset \epsilon^{-N}L_0$, which shows
\begin{equation*}
\dim\left(\frac{\epsilon^{-N}L_0}{L_0''}\right) \le \dim\left(\frac{\epsilon^{-N}L_0}{\epsilon^N L_0}\right) = 2Nr.
\end{equation*}
Therefore, $\Pi_{N,\lambda}^a$ is empty unless $a\le 2Nr$.

Now assume $0\le a\le 2Nr$. We may choose an $\epsilon$-invariant subspace $W'$ of $W = \epsilon^{-N}L_0/{\epsilon^N L_0}$ of codimension $a$. $W'$ lifts to give a $\polys$-module, say $L_0''$, with $\epsilon^N L_0\subset L_0''\subset \epsilon^{-N}L_0$ and with $\dim(\epsilon^{-N}L_0/{L_0''}) = \dim(W/W') = a$. Similarly, a flag of type $\lambda$ in $L_0''/{\epsilon L_0''}$ lifts to give $\polys$-modules $(L_{-n+1}'',\ldots,L_0'')$ with
\begin{equation*}
\epsilon L_0''\subset L_{-n+1}''\subset\cdots\subset L_{-1}''\subset L_0''\subset \epsilon^{-N}L_0
\end{equation*}
and such that the dimensions of successive quotients are given by $\lambda_1,\ldots,\lambda_n,a$, from left to right. Thus, $(L_{-n+1}'',\ldots,L_0'')$ extends by periodicity to give an element of $\Pi_{N,\lambda}^a$, as desired.
\end{proof}

{\color{blue}
\begin{lemma}
$\Pi_{N,\lambda}^a$ is a (quasi)projective variety, provided $0\le a\le 2Nr$.
\end{lemma}

\begin{proof}
Let $W= \epsilon^{-(1+N)}L_0/{\epsilon^N L_0}$ and let
\begin{equation*}
X = \left\{W_1\le\cdots\le W_n\le W:\dim\left(\frac{W}{W_n}\right)=a, \dim\left(\frac{W_i}{W_{i-1}}\right) = \lambda_i \text{ for } i=2,\ldots,n\right\}.
\end{equation*}

$X$ is known to be a projective variety {\color{red}[CITATION NEEDED]}

Let $X'$ be the subset of $X$ consisting of those $(W_1,\ldots, W_n)$, where each $W_i$ is $\epsilon$-invariant and $\epsilon W_n \le W_1$. $X'$ is a closed subset of $X$, though is not necessarily irreducible.

The correspondence between the set of $\polys$-submodules of $\epsilon^{-(1+N)}L_0$ which contain $\epsilon^N L_0$ and the set of $\polys$-submodules of $\epsilon^{-(1+N)}L_0/{\epsilon^N L_0}$ gives a pair of mutually inverse maps $\Pi_{N,\lambda}^a\leftrightarrow X'$.

 -- the idea that is relevant to the proof is that inclusion relations $L_i\subset L_{i+1}$ describe a closed set in a product of grassmanians. Unsure here -- Is it true that irreducible components of $X'$ are projective varieties. In this case, should the statement be that $\Pi_{N,\lambda}^a$ is a projective algebraic set, rather that a quasi projective variety?
\end{proof}
}

\begin{lemma}
Suppose $(L',L'')\in\mathcal{O}_B$ with $(L,L')\in\mathcal{O}_A$. Then $\xprod[L]{A,B}$ is the image of the map
\begin{equation*}
G_L\times G_{L'}\to\flags: (\alpha,\beta)\mapsto \alpha\beta L''.
\end{equation*}
\end{lemma}

\begin{proof}
Suppose $\alpha\in G_L$ and $\beta\in G_{L'}$. $(L,\alpha L',\alpha\beta L'')\in \yprod{A,B}$ since $(L,\alpha L')\sim (L,L')\in\mathcal{O}_A$ and $(\alpha L',\alpha\beta L'')\sim (L',L'')\in\mathcal{O}_B$. This shows $(L,\alpha\beta L'')\in \xprod{A,B}$ and thus $\alpha\beta L''\in \xprod[L]{A,B}$.

Conversely, suppose $N''\in \xprod[L]{A,B}$. $(L,N'')\in \xprod{A,B}$, so there is $N'$ such that $(L,N')\in\mathcal{O}_A$ and $(N',N'')\in\mathcal{O}_B$. There exist $\gamma,\delta\in G$ such that $\gamma (L,L') = (N,N')$ and $\delta (L',L'') = (N',N'')$. Then $(L,N',N'') = (L,\gamma L', \delta L'') = (L, \gamma L',\gamma (\gamma^{-1}\delta)L'')$, where $\gamma\in G_L$ and $\gamma^{-1}\delta\in G_{L'}$. This shows $N'' \in G_L G_{L'} L''$ as required.
\end{proof}

Given $N\in\naturals$, define
\begin{equation*}
H_N = \left\{ h\in G_L: h=1 \text{ on } \frac{\epsilon^{-(1+N)}L_0}{\epsilon^N L_0} \right\}.
\end{equation*}
Explicitly, the condition $h=1$ on $\epsilon^{-(1+N)}L_0/{\epsilon^N L_0}$ means: $h(x) + \epsilon^N L_0 = x + \epsilon^N L_0$ for $x\in\epsilon^{-(1+N)}L_0$. Observe that $H_{N+1}\subset H_N$ for $N\in\naturals$ since $h(x) + \epsilon^N L_0 = x + \epsilon^N L_0$ whenever $x\in\epsilon^{-(1+N)}L_0$.

\begin{lemma}
$H_N$ is a normal subgroup in $G_L$, for any $N\in\naturals$.
\end{lemma}

\begin{proof}
$H_N\subset G_L$ by definition. Suppose $h,h'\in H_N$ and let $x\in\epsilon^{-(1+N)}L_0$. $h'(x)\in\epsilon^{-(1+N)}L_0$ as $h'\in G_L$, so $hh'(x) + \epsilon^N L_0 = h'(x) + \epsilon^N L_0 = x + \epsilon^N L_0$, which shows $hh'\in H_N$. $h(x)-x\in\epsilon^N L_0$, so $h^{-1}(x) - x = -h^{-1}(h(x)-x)\in\epsilon^N L_0$. $h^{-1}\in H_N$, so $H_N$ is a subgroup of $G_L$.

Let $g\in G_L$. $hg^{-1}(x) + \epsilon^N L_0 = g^{-1}(x)$ as $g^{-1}(x)\in\epsilon^{-(1+N)}L_0$, so $ghg^{-1}(x) + \epsilon^N L_0 = gg^{-1}(x) + \epsilon^N L_0 = x + \epsilon^N L_0$. Thus $ghg^{-1}\in H_N$, which proves $H_N$ is a normal subgroup in $G_L$.
\end{proof}

The $H_N$ form a descending chain of normal subgroups in $G_L$: $\cdots\subset H_1 \subset H_0 \subset G_L \subset G$.
\begin{lemma}
$G_L/H_N$ is an irreducible algebraic group for any $N\in\naturals$.
\end{lemma}

\begin{proof}
See the discussion in \cite{lusztig99}[section 4]. Should be able to give an explicit presentation of $G_L/H_N$ in terms of the block structure.
\end{proof}

\begin{lemma}
There is $N\in\naturals$ such that $H_N\subset G_{L'}$. Consequently, $H_{N'}\subset G_{L'}$ whenever $N'\geq N$.
\end{lemma}

\begin{proof}
Choose $N\in\naturals$ such that $\epsilon^N L_0\subset L_0'\subset \epsilon^{-N}L_0$. Then
\begin{equation*}
\epsilon^N L_0 \subset L_0'\subset L_1'\subset\cdots\subset L_n' \subset \epsilon^{-(1+N)}L_0.
\end{equation*}
Let $h\in H_N$. $h(x) + \epsilon^N L_0 = x + \epsilon^N L_0$ for $x\in\epsilon^{-(1+N)} L_0$, so $h(L_i')\subset L_i'$ for $i=0,1,\ldots,n$. Moreover, $h^{-1}$ stabilises $L_i'$, so $h(L_i') = L_i'$ for $i=0,1,\ldots,n$ and therefore for $i\in\integers$. This shows $h\in G_{L'}$ as required, so $H_N\subset G_{L'}$.
\end{proof}

Note that $H_N$ is generally not a normal subgroup of $G_{L'}$, though the space of (right) cosets of $H_N$ in $G_{L'}$ will still be irreducible. {\color{blue}
ADD AN EXAMPLE}

{\color{blue}
\begin{lemma}
$G_{L'}/H_N$ is irreducible, provided $H_N\subset G_{L'}$.
\end{lemma}

\begin{proof}
\finishproof
\end{proof}
}

\begin{lemma}\label{lemma:locally-closed-orbits}
Given $L\in\flags$, the $G_L$-orbits in $\flags$ are locally closed.
\end{lemma}
\begin{proof}
{\color{red} ADD PROOF HERE.
Look at proposition 8.3 "Closed Orbits" in \cite{humphreys81}, which shows that the orbits under an algebraic group action are locally closed.}
\end{proof}

\begin{lemma}\label{lemma:irreducible-product}
Given $A,B\in\matrices$ with $\co{A}=\ro{B}$ and $L\in\flags_{\ro{A}}$, $\xprod[L]{A,B}$ is an irreducible topological space.
\end{lemma}

\section{Existence of a maximum}

\begin{lemma}\label{lemma:compare-partial-orders}
Given $A,A'\in\matrices$ with $\ro{A}=\ro{A'}$ and $\co{A}=\co{A'}$, $A'\le A$ if and only if $\xprod[L]{A'}\subset\overline{\xprod[L]{A}}$ for any $L\in\flags_{\ro{A}}$.
\end{lemma}

\begin{proof}
{\color{red}ADD PROOF}
\end{proof}

\begin{proposition}\label{proposition:existence}
Given $A,B\in\matrices$ with $\co{A}=\ro{B}$, $\matrices_{A,B}$ has a maximum element.
\end{proposition}

\begin{proof}[The Real One]
Let $L\in\flags_{\ro{A}}$. $\xprod[L]{A,B}$ is irreducible by Lemma \ref{lemma:irreducible-product} and is the union of finitely many $G_L$-orbits, namely
\begin{equation*}
\xprod[L]{A,B} = \bigcup_{C\in\matrices_{A,B}} \xprod[L]{C}.
\end{equation*}
This shows that $\xprod[L]{C}$ is dense in $\xprod[L]{A,B}$ for some $C\in\matrices_{A,B}$. Lemma \ref{lemma:locally-closed-orbits} shows that the $G_L$-orbits in $\xprod[L]{A,B}$ are locally closed, so a dense $G_L$-orbit is open in $\xprod[L]{A,B}$. Lemma \ref{lemma:compare-partial-orders} shows that the characteristic matrix of the dense $G_L$-orbit is a maximum in $\matrices_{A,B}$.
\end{proof}

{\color{gray}
\begin{proof}[Draft 1]
$\matrices_{A,B}$ is non-empty since $\co{A}=\ro{B}$. The partial order on $\matrices_{A,B}$ is given by the partial order on $\matrices$; where $C'\le C$ if and only if $d_{i,j}C'\le d_{i,j}C$ for all $i,j\in\integers$.

To prove existence of a maximum element in $\matrices_{A,B}$ we will consider the poset of $G$-orbits in $\dblflags$ and prove existence of a maximum orbit in $\xprod{A,B}$ using an open orbits argument. Recall $\xprod{A,B}$ consists of $(L,L'')\in\dblflags$ such that there exists $L'\in\flags$ with $(L,L')\in\mathcal{O}_A$ and $(L',L'')\in\mathcal{O}_B$.

There is $N\in\naturals$ such that $\epsilon^N L_0\subset L_0''\subset \epsilon^{-N}L_0$ whenever $(L,L'')\in \xprod{A,B}$. Fix $L\in\flags_{\ro{A}}$ and write
\begin{equation*}
\xprod[L]{A,B} = \{L''\in\flags: (L,L'')\in \xprod{A,B}\}.
\end{equation*}
With the above choice of $N$, write
\begin{equation*}
\Pi = \{L''\in\flags_{\co{B}}: \epsilon^N L_0\subset L_0''\subset \epsilon^{-N} L_0\}.
\end{equation*}

$\Pi$ is a complex projective variety (not generally irreducible), closed under the action of $G_L$. \needsreference The closure $\overline{\xprod[L]{A,B}}$ of $\xprod[L]{A,B}$ in $\Pi$ is an irreducible complex projective variety.

Proposition \needsreference shows there is a unique $G_L$-orbit in $\xprod[L]{A,B}$ which is open in $\overline{\xprod[L]{A,B}}$, say $\mathcal{O}_C^L$ for some $C\in\matrices_{A,B}$. It will be shown that $C$ is the maximum element of $\matrices_{A,B}$. Given $i,j\in\integers$, let $m_{i,j}$ denote the maximum of $\{d_{i,j}C: C\in\matrices_{A,B}\}$ and define
\begin{equation*}
\mathcal{M}_{i,j} = \{L''\in \overline{\xprod[L]{A,B}}: d_{i,j}(L,L'') = m_{i,j}\}.
\end{equation*}
$\mathcal{M}_{i,j}$ is non-empty by definition of the $m_{i,j}$ and is closed under the action of $G_L$. $\mathcal{M}_{i,j}$ is open in $\overline{\xprod[L]{A,B}}$ since the function
\begin{equation*}
d_{i,j}^L\colon\Pi\to\integers: L''\mapsto \dim\left(\frac{L_i}{L_i\cap L_j''}\right)
\end{equation*}
is lower semi-continuous \needsreference and
\begin{equation*}
\mathcal{M}_{i,j} = \overline{\xprod[L]{A,B}}\setminus \{L''\in\overline{\xprod[L]{A,B}}: d_{i,j}^L(L'')\le m_{i,j} -1\}.
\end{equation*}

It follows that $\mathcal{O}_C^L$ and $\mathcal{M}_{i,j}$ intersect non-trivially, since $\overline{\xprod[L]{A,B}}$ is irreducible and therefore $\mathcal{O}_C^L\subset \mathcal{M}_{i,j}$ as both are closed under the action of $G_L$. This proves $C$ is a maximum element of $\matrices_{A,B}$, since
\begin{equation*}
d_{i,j}C = d_{i,j}(L,L'') = m_{i,j}
\end{equation*}
for any $L''\in\mathcal{O}_C^L$.
\end{proof}

\begin{proof}[Draft 2]
$\matrices_{A,B}$ is non-empty since $\co{A}=\ro{B}$. For each $i,j\in\integers$, define
\begin{equation*}
m_{i,j} = \max_{C\in\matrices_{A,B}} d_{i,j}C.
\end{equation*}
It will be shown that there is a unique element $A\ast B\in \matrices_{A,B}$ with $d_{i,j}(A\ast B) = m_{i,j}$: such an element is necessarily a maximum in $\matrices_{A,B}$. Fix $L\in\flags_{\ro{A}}$ and assume $N\in\naturals$ is sufficiently large that $\xprod[L]{A,B}\subset \Pi_N$; where
\begin{equation*}
\Pi_N = \{L''\in\flags_{\co{B}}: \epsilon^N L_0\subset L_0''\subset \epsilon^{-N} L_0\}.
\end{equation*}

Lusztig notes \cite{lusztig99} that $\Pi_N$ is a projective algebraic variety, closed under the action of $G_L$. Lemma \needsreference shows that the closure of $\xprod[L]{A,B}$ in $\Pi_N$, denoted $\overline{\xprod[L]{A,B}}$, is an irreducible complex projective variety.

For each $i,j\in\integers$, write
\begin{equation*}
\mathcal{M}_{i,j} = \{L''\in\overline{\xprod[L]{A,B}}: d_{i,j}(L,L'') = m_{i,j}\}.
\end{equation*}

$\mathcal{M}_{i,j}$ is non-empty since $d_{i,j}(L,-)$ attains a maximum on $\xprod[L]{A,B}$. $\mathcal{M}_{i,j}$ is open in $\overline{\overline[L]{A,B}}$ since
\begin{equation*}
\overline{\xprod[L]{A,B}}\setminus \mathcal{M}_{i,j} = \{L''\in\overline{\xprod[L]{A,B}}: d_{i,j}(L,L'')\le m_{i,j} - 1\}
\end{equation*}
and the function
\begin{equation*}
d_{i,j}(L,-)\colon\Pi_N\to\integers : L''\mapsto \dim\left(\frac{L_i}{L_i\cap L_j''}\right)
\end{equation*}
is lower semi-continuous, by lemma [\needsreference : lower semi-continuity].

Lemma [\needsreference : open orbit] shows that there is a unique $G_L$-orbit in $\xprod[L]{A,B}$ which is open in $\overline{\xprod[L]{A,B}}$, say $\mathcal{O}_{A\ast B}^L$ for some $A\ast B\in\matrices_{A,B}$. $\mathcal{M}_{i,j}$ intersects the open orbit $\mathcal{O}_{A\ast B}^L$ non-trivially, since $\mathcal{M}_{i,j}$ and $\mathcal{O}_{A\ast B}^L$ are both non-empty and open in the irreducible space $\overline{\xprod[L]{A,B}}$. Moreover, $\mathcal{O}_{A\ast B}^L\subset \mathcal{M}_{i,j}$, since $\mathcal{M}_{i,j}$ is closed under the action of $G_L$. In particular, we have $A\ast B\in \matrices_{A,B}$ with $d_{i,j}(A\ast B) = m_{i,j}$ for each $i,j\in\integers$, which shows $A\ast B$ is a maximum in $\matrices_{A,B}$.

More specifically, we may compute:
\begin{equation*}
a_{i,j}(A\ast B) = m_{i,j-1} - m_{i-1,j-1} + m_{i-1,j} - m_{i,j}
\end{equation*}
for each $i,j\in\integers$.
\end{proof}
}

\section{Associativity}

\begin{lemma}
Suppose $(L,L',L'',L''')\in\flags^4$ with $(L,L')\in\mathcal{O}_A$, $(L',L'')\in\mathcal{O}_B$ and $(L'',L''')\in\mathcal{O}_C$. $\xprod[L]{A,B,C}$ is the image of the map
\begin{equation*}
\phi\colon G_L\times G_{L'}\times G_{L''}\to \flags: (\alpha,\beta,\gamma)\mapsto \alpha\beta\gamma L'''.
\end{equation*}
\end{lemma}

\begin{lemma}\label{lemma:irreducible-triple}
Given $A,B,C\in\matrices$ with $\co{A}=\ro{B}$ and $\co{B}=\ro{C}$ and $L\in\flags_{\ro{A}}$, $\xprod[L]{A,B,C}$ is an irreducible topological space
\end{lemma}

\begin{lemma}\label{lemma:open-embeddings}
Given $A,B,C\in\matrices$ with $\co{A}=\ro{B}$ and $\co{B}=\ro{C}$ and $L\in\flags$, $\xprod[L]{A\ast B,C}$ and $\xprod[L]{A,B\ast C}$ are open and dense in $\xprod[L]{A,B,C}$.
\end{lemma}

\begin{proposition}\label{proposition:associativity}
Given $A,B,C\in\matrices$ with $\co{A}=\ro{B}$ and $\co{B}=\ro{C}$, $(A\ast B)\ast C = A\ast (B\ast C)$.
\end{proposition}
\begin{proof}
Take $A,B,C\in\matrices$ with $\co{A}=\ro{B}$ and $\co{B}=\ro{C}$ and fix $L\in\flags_{\ro{A}}$. $\xprod[L]{A,B,C}$ is irreducible, by Lemma \ref{lemma:irreducible-triple}, and is the union of finitely many disjoint locally closed subsets, namely
\begin{equation*}
\xprod[L]{A,B,C} = \bigcup_{D\in\matrices_{A,B,C}} \xprod[L]{D}.
\end{equation*}
Therefore, exactly one of the $\xprod[L]{D}$ is open and dense in $\xprod[L]{A,B,C}$. $\xprod[L]{A\ast B,C}$ is open and dense in $\xprod[L]{A,B,C}$, by Lemma \ref{lemma:open-embeddings}. It then follows that the maximum $G_L$-orbit $\xprod[L]{(A\ast B)\ast C}$ in $\xprod[L]{A\ast B,C}$ is open and dense in $\xprod[L]{A,B,C}$. Similarly, $\xprod[L]{A\ast(B\ast C)}$ is open and dense in $\xprod[L]{A,B\ast C}$ which is in turn open and dense in $\xprod[L]{A,B,C}$. $\xprod[L]{(A\ast B)\ast C}$ and $\xprod[L]{A\ast(B\ast C)}$ are both a single orbit for the action of $G_L$ and intersect non-trivially since $\xprod[L]{A,B,C}$ is irreducible, therefore $\xprod[L]{(A\ast B)\ast C} = \xprod[L]{A\ast(B\ast C)}$, which means $(A\ast B)\ast C = A\ast(B\ast C)$.
\end{proof}


\section{The generic algebra}

\begin{lemma}\label{lemma:identity-morphisms}
Given $\lambda\in\compositions$ and $A\in\matrices$, $D_\lambda\ast A = A$ if $\ro{A}=\lambda$ and $A\ast D_\lambda = A$ if $\co{A}=\lambda$.
\end{lemma}
\begin{proof}
Lemma \ref{lemma:product-with-diagonal-orbits} shows that $\matrices_{D_\lambda,A} = \{A\}$ if $\lambda=\ro{A}$ and $\matrices_{A,D_\lambda} = \{A\}$ if $\lambda=\co{A}$, which proves the result. 
\end{proof}


\begin{definition}
For each $n,r\geq 1$, the generic category $\gcat$ is the category with set of objects $\compositions$ and set of morphisms $\matrices$ where; the morphisms from $\lambda$ to $\mu$ are those matrices $A\in\matrices$ with $\co{A}=\lambda$ and $\ro{A}=\mu$; the composition of morphisms $A\colon\lambda\to\mu$ and $B\colon\mu\to\nu$ is $B\ast A\colon\lambda\to\nu$, where $B\ast A$ is the maximum element in $\matrices_{A,B}$. For each $\lambda\in\compositions$, the identity morphism $D_\lambda\colon\lambda\to\lambda$ is given by $(D_\lambda)_{i,i}=\lambda_i$ and $(D_\lambda)_{i,j}=0$ whenever $i\neq j$.
\end{definition}

\begin{example}
The objects in $\gcat[2,2]$ are compositions of $2$ into $2$ parts, namely $(0,2)$, $(1,1)$ and $(2,0)$. The set of morphisms from $\lambda$ to $\mu$ is the set of infinite periodic matrices $A\in\matrices[2,2]$ with $\co{A}=\lambda$ and $\ro{A}=\mu$, which is a countably infinite set for any pair of compositions $\lambda,\mu\in\compositions[2,2]$.
\end{example}

\begin{definition}[Generic algebra]
The affine generic algebra $\generic$ is the category $\integers$-algebra of $\gcat$. In particular, $\generic$ is a free $\integers$-module with basis $\{e_A:A\in\matrices\}$ and with associative multiplication given by
\begin{equation*}
e_A\ast e_B = \begin{cases}
e_{A\ast B} &\text{ if } \co{A}=\ro{B}\\
0 			&\text{ if } \co{A}\neq \ro{B}.
\end{cases}
\end{equation*}
The multiplicative identity in $\generic$ is
\begin{equation*}
1 = \sum_{\lambda\in\compositions}1_\lambda
\end{equation*}
where $1_\lambda = e_{D_\lambda}$.
\end{definition}

\section{ -- Chapter draft bin -- }
{\color{gray}
Define
\begin{equation*}
\Pi = \left\{L''\in\flags_{\co{B}} : \epsilon^N L_0 \subset L_0''\subset\cdots\subset L_n''\subset \epsilon^{-N} L_0 \text{ and } \dim\left(L_0''/{\epsilon^N L_0}\right) = -Nr + d_{-Nn,0}^-(A) + d_{-Nn,0}^-(B) \right\}.
\end{equation*}

\begin{lemma}
$\Pi$ is a projective algebraic variety, closed under the action of $G_L$.
\end{lemma}

By choice of $N$, we have $\xprod[L]{A,B}\subset\Pi$.

\begin{lemma}
The group $G_L/H$ is an irreducible algebraic group.
\end{lemma}
\begin{proof}
$\sigma\in G_L$ naturally induces an automorphism $\bar{\sigma}$ of $\epsilon^{-N} L_0/{\epsilon^N L_0}$, with inverse induced by $\sigma^{-1}$. Moreover, the natural map
\begin{equation*}
G_L/H \to GL(\epsilon^{-N}L_0/{\epsilon^N L_0})
\end{equation*}
is a group homomorphism. In fact, this homomorphism is injective: if $\sigma = \tau$ on $\epsilon^{-N}L_0/{\epsilon^N L_0}$, then $\sigma\tau^{-1} = 1$ on $\epsilon^{-N}L_0/{\epsilon^N L_0}$ and so $\sigma H = \tau H$. Thus $G_L/H$ is isomorphic to its image in $GL(\epsilon^{-N}L_0/{\epsilon^N L_0})$. {\color{red}this image is an algebraic group, then I need to deduce $G_L/H$ is an algebraic group. First isomorphism theorem?} 
\end{proof}

\begin{lemma}
Suppose $(L,L',L''),(N,N',N'')\in\beta^{-1}(\mathcal{O}_A\times\mathcal{O}_B)$. Then there are $\sigma,\tau\in G$, with $\tau\in G_{L'}$, such that $(N,N',N'') = \sigma(L,L',\tau L'')$.
\end{lemma}
\begin{proof}
There exist $g,g'\in G$ such that $(N,N') = g(L,L')$ and $(N',N'') = g'(L',L'')$. Then $(N,N',N'') = g(L,L',g^{-1}g' L'')$. Taking $\sigma = g$ and $\tau = g^{-1}g'$ gives the required result.
\end{proof}

\begin{proposition}
Let $A,B\in\matrices$, $L\in\flags$ and suppose $\xprod[L]{A,B}\neq\emptyset$. There is a unique open $G_L$-orbit in $\xprod[L]{A,B}$.
\end{proposition}

\begin{proof}\color{gray}
Write $X=\xprod[L]{A,B}$. $X$ is irreducible and finite dimensional, using Lemma  \ref{lemma:irreducible-product}. We have
\begin{equation*}
X = \bigcup_C O_C,
\end{equation*}
where the union is taken over the finite set $\{C\in\matrices : \mathcal{O}_C\subset\xprod{A,B}\}$.

A proper, non-empty, closed subset of $X$ has strictly smaller dimension than $X$, so there is $C$ such that $\overline{O_C}=X$. $O_C$ is locally closed, by Lemma \ref{lemma:locally-closed-orbits}, so it follows that $O_C$ is open in $\overline{O_C}=X$.

Now suppose $O_C$ is an open $G_L$ orbit and let $D\in\matrices$. $O_D\subset X\setminus O_C$ and thus $\overline{O_D}\subset X\setminus O_C$. This shows $O_D$ is not open in $X$ and thus the claim is proven.
\end{proof}
}

\chapter{A realisation of affine zero Schur algebras}

We aim to prove the isomorphism theorem in the cases $r<n$ and $n\le r< 2n$ separately. Below are crude versions of the statements we want to prove.

\begin{theorem}
Assume $r<n$. The map $\psi\colon\generic\to\zeroschur$, given by $\psi(E_i)=E_i$, $\psi(F_i
)=F_i$ and $\psi(1_\lambda) = 1_\lambda$, is an isomorphism of $\integers$-algebras.
\end{theorem}
\begin{proof}{\color{gray}
Below are some of the pieces:
[1] The elements $E_i$, $F_i$, $1_\lambda$ generate $\generic$.

Provided $r<n$, any $A\in\matrices$ may be obtained from the diagonal matrix $D_\lambda$ with $\lambda=\ro{A}$ by a sequence of transitions $A \mapsto A\pm X_{i,p}$.

[2] Give a complete set of generating relations for $\generic$.}
\end{proof}

\begin{theorem}
Assume $n\le r< 2n$. There is a unique homomorphism of $\integers$-algebras $\hat{\psi}\colon\generic\to\zeroschur$ such that $\hat{\psi}(R)=R$ and $\hat{\psi}=\psi$ on the subalgebra of $\generic$ generated by the $E_i$, $F_i$ and $1_\lambda$. $\hat{\psi}$ is an isomorphism of $\integers$-algebras.
\end{theorem}


\section{Multiplication rules}

Write
\begin{equation*}
E_i = \sum_{\lambda\in\compositions} E_{i,\lambda}
\end{equation*}
\begin{equation*}
F_i = \sum_{\lambda\in\compositions} F_{i,\lambda}.
\end{equation*}
Then $E_{i,\lambda} = E_i 1_\lambda$ and $F_{i,\lambda} = F_i 1_\lambda$.

\begin{lemma}
Let $A\in\matrices$, $i\in [1,n]$ and let $\lambda = \ro{A}$. The following multiplication rules hold:
\begin{equation*}
E_i e_A = \begin{cases}
e_{A+X_{i,p}} &\text{	if } \lambda_{i+1}>0\\
0 &\text{	if } \lambda_{i+1}=0;
\end{cases}
\end{equation*}
where $p$ is such that $A_{i+1,p}>0$ and $A_{i+1,j}=0$ for $j>p$. Also
\begin{equation*}
F_i e_A = \begin{cases}
e_{A-X_{i,p}} &\text{	if } \lambda_i>0\\
0 &\text{	if } \lambda_i=0;
\end{cases}
\end{equation*}
where $p$ is such that $A_{i,p}>0$ and $A_{i,j}=0$ for $j<p$.
\end{lemma}

Similar formulas for right multiplication by $E_i$ and $F_i$ are obtained by applying the transpose.

\begin{lemma}
The following relations hold in $\generic$ ($n\geq 3$):
\begin{equation*}
E_iE_j - E_jE_i = 0
\end{equation*}
\begin{equation*}
F_iF_j - F_jF_i = 0
\end{equation*}
unless $|j-i|=1$.
\begin{equation*}
E_iE_{i+1}^2 - E_{i+1}E_iE_{i+1} = 0
\end{equation*}
\begin{equation*}
E_i^2E_{i+1} - E_iE_{i+1}E_i = 0
\end{equation*}
\begin{equation*}
F_{i+1}F_i^2 - F_iF_{i+1}F_i = 0
\end{equation*}
\begin{equation*}
F_{i+1}^2F_i - F_{i+1}F_iF_{i+1} = 0
\end{equation*}

\begin{equation*}
E_iF_j - F_jE_i = 0
\end{equation*}
unless $j=i$.
\begin{equation*}
E_iFi - F_iE_i + \sum_{\lambda:\lambda_i = 0,\lambda_{i+1}>0} 1_\lambda - \sum_{\lambda:\lambda_i>0, \lambda_{i+1}=0} 1_\lambda = 0.
\end{equation*}
\end{lemma}


\section{Presentation of the generic algebra.}

Recall that $\compositions$ denotes the set of compositions of $r$ into $n$ parts. That is, $\compositions$ is the set of tuples $\lambda = (\lambda_1,\ldots,\lambda_n)\in\integers^n$ with each $\lambda_i$ non-negative and $\lambda_1 +\cdots +\lambda_n = r$. Given $i\in [1,n]$, let $\epsilon_i = (0,\ldots,1,\ldots,0)\in\integers^n$ be the $i$--th elementary vector and let $\alpha_i = \epsilon_i - \epsilon_{i+1}$. Then given $\lambda\in\compositions$, we have $\lambda + \alpha_i\in\compositions$ provided $\lambda_{i+1}>0$ and $\lambda - \alpha_i\in\compositions$ provided $\lambda_i>0$.

%define the quiver:
Let $\Gamma =\presentationquiver$ be the quiver with set of vertices $\compositions$ with arrows $e_{i,\lambda}\colon\lambda\to\lambda +\alpha_i$ (if $\lambda_{i+1}>0$) and $f_{i,\lambda}\colon\lambda\to\lambda -\alpha_i$ (if $\lambda_i>0$). Thus there are no arrows between $\lambda$ and $\mu$ unless $\lambda = \mu\pm \alpha_i$ for some $i\in [1,n]$.

If $n\geq 3$ then neighbouring vertices are connected by two arrows, one of each direction. In the case $n=2$, neighbouring vertices are joined by four arrows, two of each direction. The $\integers\Gamma$ denote the path $\integers$ algebra of $\Gamma$. By construction of $\Gamma$, there is a $\integers$-algebra homomorphism $\integers\Gamma\to\generic$ with $e_{i,\lambda}\mapsto E_{i,\lambda}$, $f_{i,\lambda}\mapsto F_{i,\lambda}$ and $k_\lambda = 1_\lambda$. We aim to describe the image and kernel of the morphism to give a presentation of the generic algebra by a quiver with relations, when possible. In general, we should obtain a presentation of a subalgebra of the generic algebra consisting of the so-called aperiodic elements (c.f. \cite{lusztig99}).

% defining the set of aperiodic elements.

$A\in\matrices$ is said to be aperiodic if for each $l\in\integers\setminus\{0\}$ there exists $i\in\integers$ such that $a_{i,i+l}=0$. Denote the set of aperiodic elements in $\matrices$ by $\matrices^{ap}$. Note that $\matrices^{ap}=\matrices$ if $r<n$.


% image spanned by aperiodic elements. for all n.
\begin{proposition}
The subalgebra of $\generic$ generated by $E_{i,\lambda}$, $F_{i,\lambda}$ and $1_\lambda$ has $\integers$-basis $\{e_A:A\in\matrices^{ap}\}$, where $\matrices^{ap}\subset\matrices$ is the set of aperiodic elements.
\end{proposition}


% comment that all elements are aperiodic in the case r<n.

% Relations in the n>3 case

% Relations in the n=2 case.



\chapter{Further directions}

\section{Further results on affine zero Schur algebras}

[1] Investigate link between this generic product and the generic extension of representations. Shifting to the non-negative subalgebra to do computations purely in terms of generic extensions of quiver representations.

\section{Deformed group algebras of symmetric groups}

[2] Degenerate group algebras of symmetric groups: write down a presentation of the degenerate group algebras, with generators given by the transpositions, or $2$-cycles. Type up the computations done for degenerate group algebras for $S_3$ and $S_4$. Formulate propositions for the general case: the transpositions generate the degenerate group algebra; lemma: `these' relations hold; these generators and relations give a presentation of the degenerate group algebras.

Terminology: deformed group algebra.

% --- END OF DOCUMENT BODY --- %

\printbibliography

\end{document}
